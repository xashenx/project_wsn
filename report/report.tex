%%%%%%%%%%%%%%%%%%%%%%%%%%%%%%%%%%%%%%%%%
% University/School Laboratory Report
% LaTeX Template
% Version 3.0 (4/2/13)
%
% This template has been downloaded from:
% http://www.LaTeXTemplates.com
%
% Original author:
% Linux and Unix Users Group at Virginia Tech Wiki 
% (https://vtluug.org/wiki/Example_LaTeX_chem_lab_report)
%
% License:
% CC BY-NC-SA 3.0 (http://creativecommons.org/licenses/by-nc-sa/3.0/)
%
%%%%%%%%%%%%%%%%%%%%%%%%%%%%%%%%%%%%%%%%%

%----------------------------------------------------------------------------------------
%	PACKAGES AND DOCUMENT CONFIGURATIONS
%----------------------------------------------------------------------------------------

\documentclass{article}

\usepackage{graphicx} % Required for the inclusion of images
\usepackage{listings} % Required for insertion of code
\setlength\parindent{0pt} % Removes all indentation from paragraphs
\usepackage[usenames,dvipsnames]{color} % Required for custom colors
\renewcommand{\labelenumi}{\alph{enumi}.} % Make numbering in the enumerate environment by letter rather than number (e.g. section 6)
\definecolor{mygreen}{rgb}{0,0.6,0}
\definecolor{mygray}{rgb}{0.5,0.5,0.5}
\definecolor{mymauve}{rgb}{0.58,0,0.82}
%\usepackage{times} % Uncomment to use the Times New Roman font
\usepackage{graphicx, array, blindtext}
%----------------------------------------------------------------------------------------
%	CODE INCLUSION CONFIGURATION
%----------------------------------------------------------------------------------------

\definecolor{MyDarkGreen}{rgb}{0.0,0.4,0.0} % This is the color used for comments
\lstloadlanguages{C} % load C syntax

\lstset{ %
  backgroundcolor=\color{SpringGreen},   % choose the background color; you must add \usepackage{color} or \usepackage{xcolor}
  basicstyle=\footnotesize,        % the size of the fonts that are used for the code
  breakatwhitespace=false,         % sets if automatic breaks should only happen at whitespace
  breaklines=true,                 % sets automatic line breaking
  captionpos=b,                    % sets the caption-position to bottom
  commentstyle=\color{mygreen},    % comment style
  deletekeywords={...},            % if you want to delete keywords from the given language
  escapeinside={\%*}{*)},          % if you want to add LaTeX within your code
  extendedchars=true,              % lets you use non-ASCII characters; for 8-bits encodings only, does not work with UTF-8
  frame=single,                    % adds a frame around the code
  keywordstyle=\color{blue},       % keyword style
  language=C,					   % the language of the code
  morekeywords={*,...},            % if you want to add more keywords to the set
  numbers=left,                    % where to put the line-numbers; possible values are (none, left, right)
  numbersep=5pt,                   % how far the line-numbers are from the code
  stepnumber=3,    
  firstnumber=1,
  numberfirstline=false
  %numberstyle=\tiny\color{gray}, % the style that is used for the line-numbers
  rulecolor=\color{black},         % if not set, the frame-color may be changed on line-breaks within not-black text (e.g. comments (green here))
  showspaces=false,                % show spaces everywhere adding particular underscores; it overrides 'showstringspaces'
  showstringspaces=false,          % underline spaces within strings only
  showtabs=false,                  % show tabs within strings adding particular underscores
  %stepnumber=2,                    % the step between two line-numbers. If it's 1, each line will be numbered
  stringstyle=\color{mymauve},     % string literal style
  tabsize=2,                       % sets default tabsize to 2 spaces
  title=\lstname                   % show the filename of files included with \lstinputlisting; also try caption instead of title
}

%----------------------------------------------------------------------------------------
%	DOCUMENT INFORMATION
%----------------------------------------------------------------------------------------

%----------------------------------------------------------------------------------------
%	TITLE PAGE
%----------------------------------------------------------------------------------------

\title{
\vspace{2in}
\textmd{\textbf{Load Balanced Routing Protocol}}\\
\textmd{\normalsize{Wireless Sensor Networks}}\\
%\normalsize\vspace{0.1in}\small{Due\ on\ \hmwkDueDate}\\
%\vspace{0.1in}\large{\textit{\hmwkClassInstructor\ \hmwkClassTime}}
\vspace{3in}
}

\author{\textbf{Fabrizio Zeni} \\ \small{153465}}
\date{} % Insert date here if you want it to appear below your name

\begin{document}

\maketitle % Insert the title, author and date
\thispagestyle{empty}
\begin{center}
\begin{tabular}{l r}
\end{tabular}
\end{center}
\clearpage
% If you wish to include an abstract, uncomment the lines below
\thispagestyle{empty}
\begin{abstract}
This report describes the project developed for the wireless sensor networks laboratory.
It starts from the project specifications and continues through the implementation choices.
It ends with the experimental results and their analysis.
\end{abstract}
\tableofcontents
\clearpage
%----------------------------------------------------------------------------------------
%	SECTION 1 - PROJECT SPECIFICATION
%----------------------------------------------------------------------------------------
\pagenumbering{arabic}
\section{Introduction}
\subsection{Project Specifications}
The project specifications ask to implement a \emph{load balanced routing protocol} over \textit{TOSSIM} which is capable to route the data traffic through a wireless sensor network up to a node called \textbf{sink}, which is responsible of the data collection.
Each node is responsible to mantain its own "routing table", which in this case is represented by a list of \emph{parent nodes} as the next-hop node toward the sink. To balance the traffic among the parents, the node has to check the amount of forwarded message of each parent and verify, for each pair of parents, the following formula: \begin{center}$0 \leq msg_{p_{a}} - msg_{p_{b}} \leq 1$\end{center}
where $msg_{p_{i}}$ are the messages forwarded to the parent $p_{i}$ of the node.
\subsection{How to run the project}
Looking at the project root, the code is contained mainly in two directories: \textbf{Graph} and \textbf{Tree}. The implementation of the LBRP is in the Graph one, while the other contain the \textit{single-parent} implementation, which was used to make some comparison over the developed protocol. The application is \textit{modular}, in the sense that from the same code is possible to compile each subversion of the application. That is made possible through some \emph{preprocessor conditional branches} inside the code (\textit{\#ifdef keyword}), which are triggered when the compilation is done using a specific makefile. These are the makefiles defined so far:
\begin{itemize}
  \item Makefile - the default one, which compiles the standard version of the application
  \item Reliable - which compiles the reliable 1-1 communications version
  \item Remove - which compiles the Cascade Parent Removal version
  \item Debug - which compiles the debug version of the program
\end{itemize}

If the version to compile is the standard one, it will be suffies to run "\textit{make micaz sim}", else the command to input will be "\textit{make -f \textbf{makefile} micaz sim}", where \textbf{makefile} is the makefile of the version to compile.\\
\\Then to run the compiled program, one of the python scripts has to be chosen. 
\begin{itemize}
  \item generated.py - this script will execute the script over a topology generated through the topology generator contained in tinyos
  \item newlink.py - executes the script over an ad-hoc scenario, built to check the response of the protocol
  \item test.py - as for the previous, executes over an another ad-hoc scenario
\end{itemize}
To run the simulation type "\textit{python name\_of\_script}" and the simulation will begin.
To simplify this procedure, I prepared a \emph{bash script}, which is described in section~\ref{sec:bash}.

\clearpage
%----------------------------------------------------------------------------------------
%	SECTION 1 - PROJECT IMPLEMENTATION
%----------------------------------------------------------------------------------------

\section{Project Implementation}
\subsection{Bird's-eye overview}
\subsection{Features}
\subsubsection{Best-Effort 1-1}
\subsubsection{Reliable 1-1}
\subsubsection{CPR - Cascade Parent Removal}
\subsection{Tools}
\subsubsection{Bash script}
\label{sec:bash}
\clearpage
%----------------------------------------------------------------------------------------
%	SECTION 2 - Experimental results
%----------------------------------------------------------------------------------------

\section{Experimental Results}
\subsection{Best-Effort 1-1}
\subsection{Reliable 1-1}
\subsection{CPP - Cascade Parent Removal}

%----------------------------------------------------------------------------------------
%	SECTION 4
%----------------------------------------------------------------------------------------

\section{Results and Conclusions}

The atomic weight of magnesium is concluded to be, as determined by the stoichiometry of its chemical combination with oxygen. This result is in agreement with the accepted value.

\begin{figure}[htbp]
\begin{center}
\includegraphics[scale=.35]{graphs/newlink_0_start} % Include the image placeholder.png
\caption{Newlink starting scenario}
\end{center}
\end{figure}

\begin{table}[ht]
\caption{A table arranging  images}
\centering
\begin{tabular}{cc}
\includegraphics[scale=.45]{graphs/newlink_0_start}&\includegraphics[scale=.35]{graphs/newlink_1_routing}\\
 
\includegraphics[scale=.45]{graphs/newlink_2_remove}&\includegraphics[scale=.45]{graphs/newlink_3_add1}\\
\end{tabular}
\label{tab:gqwt}
\end{table}


%----------------------------------------------------------------------------------------
%	APPENDIXES
%----------------------------------------------------------------------------------------

\begin{table}[ht]
\caption{Newlink example}
\centering
\begin{tabular}{*{2}{m{0.48\textwidth}}}
\hline
This is some text&\begin{center}\includegraphics[scale=.45]{graphs/newlink_0_start}\end{center}\\
\hline
\blindtext&\begin{center}\includegraphics[scale=.45]{graphs/newlink_1_routing}\end{center}\\
\hline
\end{tabular}
\label{tab:gAt}
\end{table}
\clearpage
\begin{table}[ht]
\caption{Test example}
\centering
\begin{tabular}{*{2}{m{0.48\textwidth}}}
\hline
This is some text&\begin{center}\includegraphics[scale=.45]{graphs/newlink_0_start}\end{center}\\
\hline
\blindtext&\begin{center}\includegraphics[scale=.45]{graphs/newlink_1_routing}\end{center}\\
\hline
\end{tabular}
\label{tab:gAasdt}
\end{table}
\clearpage

%----------------------------------------------------------------------------------------
%	BIBLIOGRAPHY
%----------------------------------------------------------------------------------------

\bibliographystyle{unsrt}

\bibliography{sample}

%----------------------------------------------------------------------------------------


\end{document}